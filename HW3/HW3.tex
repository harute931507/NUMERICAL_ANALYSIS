\documentclass[12pt]{article}
\usepackage{amssymb}
\usepackage{enumerate}
\usepackage{commath}
\usepackage{fontspec} 
\usepackage{xeCJK}
\usepackage[left=2cm,right=2cm,top=2cm,bottom=2cm]{geometry}
\usepackage{amsmath}
\usepackage{listings}
\parindent=0pt
\setCJKmainfont{標楷體} 
\XeTeXlinebreaklocale "zh"
\XeTeXlinebreakskip = 0pt plus 1pt
\title{數值分析Team5 Homwork3}
\author{盧勁綸\and 張毓軒\and 李奇軒\and 王宥鈞}
\date{}


\begin{document}
\maketitle

[Theoretical problems]
\begin{enumerate}
    \item
        \begin{enumerate}
            \item %1(a) 
            Use Gaussian quadrature with $n = 2$~,~let $f(x) = \frac{2x}{x^2-4}$
            \begin{eqnarray*}
                \int_{1}^{1.6} \frac{2x}{x^2-4}~dx &=& \frac{1.6-1}{2}\int_{-1}^{1}f\left(\frac{1.6-1}{2}x+\frac{1.6+1}{2}\right)~dx \\
                &=& 0.3\int_{-1}^{1}f(0.3x+1.3)~dx\\
                & \approx & 0.3~[~\alpha_1 f(x_1)+\alpha_2 f(x_2)+\alpha_3 f(x_3)~]               
            \end{eqnarray*} 
            \begin{align*}
            \text{where}~~\alpha_1 & = \frac{5}{9}\approx 0.5556~,~x_1=-\sqrt{0.6}\approx-0.7746\\
            \alpha_2&=\frac{8}{9}\approx0.8889~,~x_2=0\\
            \alpha_3&=\frac{5}{9}\approx0.5556~,~x_3=\sqrt{0.6}\approx0.7746     
            \end{align*}
            \begin{eqnarray*}
            \Rightarrow \int_{1}^{1.6}\frac{2x}{x^2-4}~dx &\approx& 0.3~[~0.5556*f(~0.3(-0.7746)+1.3~)+0.8889*f(~0.3*0+1.3~)\\
            &&+0.5556*f(~0.3(0.7746)+1.3~)~]\\
            &=&0.3~[~0.556*f(1.0676)+0.8889*f(1.3)+0.5556*f(1.5324)~]\\
            &=&0.3~[~(-0.4427)+(-1.0005)+(-1.0309)~]\\
            &=&-0.7422
            \end{eqnarray*}
            And the exact value is
            \begin{eqnarray*}
            \int_{1}^{1.6}\frac{2x}{x^2-4}~dx&=&\int_{3}^{1.44}\frac{dy}{y}
            ~~~\text{(let $y=4-x^2 , dy=-2xdx$)}\\
            &=&[~\ln(y)~]_{3}^{1.44}\\
            &=&\ln(1.44)-\ln(3)\\
            &=&-0.734
            \end{eqnarray*}
     
            \item %1(b)
            Use Gaussian quadrature with $n = 2$~,~let $f(x) = \frac{x}{\sqrt{x^2-4}}$
            \begin{eqnarray*}
            \int_{3}^{3.5}\frac{x}{\sqrt{x^2-4}}&=&\frac{3.5-3}{2}\int_{-1}^{1}f\left(\frac{3.5-3}{2}x+\frac{3.5+3}{2}\right)~dx\\
            &\approx& 0.25~[~\alpha_1 f(x_1)+\alpha_2 f(x_2)+\alpha_3 f(x_3)~]\\
            &=&0.25~[~0.5556*f(~0.25*(-0.7746)+3.25~)+0.8889*f(3.25)\\
            &&+0.5556*f(~0.25*(0.7746)+3.25~)~]\\
            &=&0.25~[~0.5556*f(3.0564)+0.8889*f(3.25)+0.5556*f(3.4437)~]\\
            &=&0.25~[~0.7345+1.1277+0.6825~]\\
            &=&0.6362
            \end{eqnarray*}
            And the exact value is
            \begin{eqnarray*}
            \int_{3}^{3	.5}\frac{x}{\sqrt{x^2-4}}~dx&=&\int_{5}^{8.25}\frac{dy}{2\sqrt{y}}
            ~~~\text{(let $y=x^2-4 , dy=-2xdx$)}\\
            &=&[~\sqrt{y}~]_{5}^{8.25}\\
            &=&\sqrt{8.25}-\sqrt{5}\\
            &=&2.8723-2.2361\\
            &=&0.6362
            \end{eqnarray*}            
        \end{enumerate}
	\item 
	We know that given $n$ point, we can get the order of precision to $2n-1$ \\\\
	then consider the following function. \\
	\begin{eqnarray*}
	f(x)={\Pi_{i=1}^{n}}(x-x_i)^2
	\end{eqnarray*}
	then by the lecture note
	\begin{eqnarray*}
	Q_{n}(x)&=&\Sigma_{i=1}^{n}w_{i}f(x_i)=0 \\
	I(f)&=&\int_{a}^{b}f(x)~dx>0 \\
	E_{n}&=&I(f)-Q_{n}>0\\
	\end{eqnarray*}
	By expanding $f$, we knew that $f$ can be express as
	\begin{eqnarray*}
    f(x)=\Sigma_{i=0}^{2n}~\alpha_{i}x^{i}
	\end{eqnarray*}
	for each $x^{i}$ , $0 \leq i \leq 2n-1$
	\begin{eqnarray*}
	E_{n}(x^i)&=&I(x^i)-Q_{n}(x^i)=0\\
	\Rightarrow E_{n}(\alpha_{i}x^i)&=&I(\alpha_{i}x^i)-Q_{n}(\alpha_{i}x^i)=0\\	
	\text{but } E_{n}&=&I(f)-Q_{n}>0\\
	\Rightarrow \ E_{n}(x^{2n})&=&I(x^{2n})-Q_{n}(x^{2n})>0 ~~~~QED
	\end{eqnarray*}
	
	
	\item
	Let $h=\frac{b-a}{3}$~,~$x_0=a+h$~,~$x_1=b+h$\\\\
	By Newton formula\\
	\begin{eqnarray*}
	f(t)&=&f(x_0)\frac{t-x_1}{x_0-x_1}+f(x_1)\frac{t-x_0}{x_1-x_0}\\
	\widetilde{\int_{a}^{b}}f(t)~dt&=&\int_{a}^{b}f(x_0)\frac{t-x_1}{x_0-x_1}+f(x_1)\frac{t-x_0}{x_1-x_0}~dt\\
	&=&f(\frac{2a+b}{3})\frac{\frac{b^2-a^2}{2}-\frac{a+2b}{3}(b-a)}{\frac{a-b}{3}}
	+f(\frac{a+2b}{3})\frac{\frac{b^2-a^2}{2}-\frac{2a+b}{3}(b-a)}{\frac{b-a}{3}}\\
	&=&\frac{3}{2}(\frac{b-a}{3})\left[f\left(\frac{2a+b}{3}\right)+f\left(\frac{a+2b}{3}\right)\right]\\
	&=&\frac{b-a}{2}\left[f\left(\frac{2a+b}{3}\right)+f\left(\frac{a+2b}{3}\right)\right]\\
	\end{eqnarray*}
	\begin{enumerate}[{(1)}]
		\item If $f(x)=1~,~\int_{a}^{b}1~dx=b-a$
			\begin{eqnarray*}
				\widetilde{\int_{a}^{b}}1~dx=\frac{b-a}{2}[1+1]=b-a
			\end{eqnarray*}
		\item If $f(x)=x$~,~$\int_{a}^{b}x~dx=\frac{b^2-a^2}{2}$
			\begin{eqnarray*}
				\widetilde{\int_{a}^{b}}x~dx=\frac{b-a}{2}(\frac{3a+3b}{3})=						\frac{b^2-a^2}{2}\\
			\end{eqnarray*}
		\item If $f(x)=x^2~,~\int_{a}^{b}x^2~dx = \frac{b^3-a^3}{3}$
			\begin{eqnarray*}
				\widetilde{\int_{a}^{b}}x^2~dx &=&\frac{b-a}{2}\left(\frac{4a^2+4ab+b^2}{9}+\frac{a^2+4ab+4b^2}{9}\right)\\
				&=&\frac{(b-a)(5a^2+8ab+5b^2)}{18}\\
				&\neq&\frac{b^3-a^3}{3}
			\end{eqnarray*}
	\end{enumerate}

        
\end{enumerate}

[Numerical Problems]
\begin{enumerate}
    \item
    \lstset{language=Matlab,showstringspaces=false}
	\lstinputlisting{inte.m}
\end{enumerate}
\end{document}
