\documentclass[12pt]{article} 
\usepackage{fontspec} 
\usepackage{xeCJK}
\usepackage[left=2cm,right=2cm,top=2cm,bottom=2cm]{geometry}
\usepackage{amsmath}
\usepackage{listings}
\parindent=0pt
\setCJKmainfont{標楷體} 
\XeTeXlinebreaklocale "zh"
\XeTeXlinebreakskip = 0pt plus 1pt
\title{數值分析Team5 Homwork2 v2}
\author{盧勁綸\and 張毓軒\and 李奇軒\and 王宥鈞}
\date{}

\begin{document}
\maketitle
[Theoretical problems]
\begin{enumerate}
\item 
略
\item 
Let $(0.1)_{10}=(0.a_1a_2a_3a_4a_5...)_{2}$ 
\begin{eqnarray*}
(0.2)_{10}&=&(a_1.a_2a_3a_4...)_2 \Rightarrow a_1=0\\
(0.4)_{10}&=&(a_2.a_3a_4a_5...)_2 \Rightarrow a_2=0\\
(0.8)_{10}&=&(a_3.a_4a_5a_6...)_2 \Rightarrow a_3=0\\
(1.6)_{10}&=&(a_4.a_5a_6a_7...)_2 \Rightarrow a_4=1\\
(0.6)_{10}&=&(0.a_5a_6a_7...)_2\\
(1.2)_{10}&=&(a_5.a_6a_7a_8...)_2 \Rightarrow a_5=1\\
(0.2)_{10}&=&(0.a_6a_7a_8...)_2\\
(0.4)_{10}&=&(a_6.a_7a_8a_9...)_2 \Rightarrow a_6=0
\end{eqnarray*}	
we get the relations
\begin{eqnarray*}
a_1&=&0\\
a_2&=&a_6\ =\ a_{2+4k}\ =\ 0\\
a_3&=&a_7\ =\ a_{3+4k}\ =\ 0\\
a_4&=&a_8\ =\ a_{4k}\ \ \ \ =\ 1\\
a_5&=&a_9\ =\ a_{1+4k}\ =\ 1
\end{eqnarray*}
therefore
\begin{eqnarray*}
(0.1)_{10}&=&(0.000110011001100...)_{2}\\
&=&(-1)^{0}*2^{-4}*(1+0.100110011001100110011001...)_2\\\\
sign\ s&=&0\\
exponent\ e&=&(-4+127)=123=(01111011)_{2}\\
mantissa\ m&=&10011001100110011001101 
\end{eqnarray*}
So 0.1's floating-point number is 0 01111011 10011001100110011001101


\item 
\begin{enumerate}

\item 
略\\

\item
\begin{eqnarray*}
b^2&=&11.1556\\
4ac&=&11.1264\\
b^2-4ac&=&0.0292\\
\end{eqnarray*}


\item
\begin{eqnarray*}
\frac{\left|0.0292-0.1\right|}{0.1}&=&\frac{0.0708}{0.1}
\ =\ 0.708
\end{eqnarray*}
\end{enumerate}
\end{enumerate}


[Numerical Problems]
\begin{enumerate}

\item
The following is our MATLAB code\\
\lstset{language=Matlab,showstringspaces=false}
\lstinputlisting{findTheNumber.m}
\begin{eqnarray*}
\epsilon_1 = 2^{-53} \ , \epsilon_2 = 2^{-1075}
\end{eqnarray*}

These two numbers are not the same , 
we guess the motion $+1$ involved in the arithmetic of floating point numbers, makes the accuracy decrease.\\

\item
\begin{enumerate}
\item
略
\item
略
\end{enumerate}
\end{enumerate}
\end{document}